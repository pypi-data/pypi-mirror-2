% Generated by Sphinx.
\documentclass[a4paper,10pt,english]{manual}
\usepackage[utf8]{inputenc}
\usepackage[T1]{fontenc}
\usepackage{babel}
\usepackage{times}
\usepackage[Bjarne]{fncychap}
\usepackage{longtable}
\usepackage{sphinx}


\title{Baruwa Documentation}
\date{April 02, 2010}
\release{0.0.1a}
\author{Andrew Colin Kissa}
\newcommand{\sphinxlogo}{}
\renewcommand{\releasename}{Release}
\makeindex
\makemodindex
\newcommand\PYGZat{@}
\newcommand\PYGZlb{[}
\newcommand\PYGZrb{]}
\newcommand\PYGaz[1]{\textcolor[rgb]{0.00,0.63,0.00}{#1}}
\newcommand\PYGax[1]{\textcolor[rgb]{0.84,0.33,0.22}{\textbf{#1}}}
\newcommand\PYGay[1]{\textcolor[rgb]{0.00,0.44,0.13}{\textbf{#1}}}
\newcommand\PYGar[1]{\textcolor[rgb]{0.73,0.38,0.84}{#1}}
\newcommand\PYGas[1]{\textcolor[rgb]{0.25,0.44,0.63}{\textit{#1}}}
\newcommand\PYGap[1]{\textcolor[rgb]{0.78,0.36,0.04}{#1}}
\newcommand\PYGaq[1]{\textcolor[rgb]{0.38,0.68,0.84}{#1}}
\newcommand\PYGav[1]{\textcolor[rgb]{0.00,0.44,0.13}{\textbf{#1}}}
\newcommand\PYGaw[1]{\textcolor[rgb]{0.13,0.50,0.31}{#1}}
\newcommand\PYGat[1]{\textcolor[rgb]{0.32,0.47,0.09}{#1}}
\newcommand\PYGau[1]{\textcolor[rgb]{0.13,0.50,0.31}{#1}}
\newcommand\PYGaj[1]{\textcolor[rgb]{0.00,0.44,0.13}{#1}}
\newcommand\PYGak[1]{\textcolor[rgb]{0.14,0.33,0.53}{#1}}
\newcommand\PYGah[1]{\textcolor[rgb]{0.00,0.13,0.44}{\textbf{#1}}}
\newcommand\PYGai[1]{\textcolor[rgb]{0.73,0.38,0.84}{#1}}
\newcommand\PYGan[1]{\textcolor[rgb]{0.00,0.44,0.13}{\textbf{#1}}}
\newcommand\PYGao[1]{\textcolor[rgb]{0.25,0.44,0.63}{\textbf{#1}}}
\newcommand\PYGal[1]{\colorbox[rgb]{1.00,0.94,0.94}{\textcolor[rgb]{0.25,0.50,0.56}{#1}}}
\newcommand\PYGam[1]{\textbf{#1}}
\newcommand\PYGab[1]{\textit{#1}}
\newcommand\PYGac[1]{\textcolor[rgb]{0.25,0.44,0.63}{#1}}
\newcommand\PYGaa[1]{\textcolor[rgb]{0.19,0.19,0.19}{#1}}
\newcommand\PYGaf[1]{\textcolor[rgb]{0.25,0.50,0.56}{\textit{#1}}}
\newcommand\PYGag[1]{\textcolor[rgb]{0.13,0.50,0.31}{#1}}
\newcommand\PYGad[1]{\textcolor[rgb]{0.25,0.44,0.63}{#1}}
\newcommand\PYGae[1]{\textcolor[rgb]{0.13,0.50,0.31}{#1}}
\newcommand\PYGaZ[1]{\textcolor[rgb]{0.02,0.16,0.45}{\textbf{#1}}}
\newcommand\PYGbf[1]{\textcolor[rgb]{0.40,0.40,0.40}{#1}}
\newcommand\PYGaX[1]{\textcolor[rgb]{0.00,0.44,0.13}{#1}}
\newcommand\PYGaY[1]{\textcolor[rgb]{0.25,0.44,0.63}{#1}}
\newcommand\PYGbc[1]{\textcolor[rgb]{0.00,0.44,0.13}{\textbf{#1}}}
\newcommand\PYGbb[1]{\textcolor[rgb]{0.13,0.50,0.31}{#1}}
\newcommand\PYGba[1]{\textcolor[rgb]{0.00,0.00,0.50}{\textbf{#1}}}
\newcommand\PYGaR[1]{\textcolor[rgb]{0.73,0.38,0.84}{#1}}
\newcommand\PYGaS[1]{\textcolor[rgb]{0.25,0.50,0.56}{\textit{#1}}}
\newcommand\PYGaP[1]{\textcolor[rgb]{0.25,0.44,0.63}{#1}}
\newcommand\PYGaQ[1]{\textcolor[rgb]{0.13,0.50,0.31}{#1}}
\newcommand\PYGaV[1]{\textcolor[rgb]{0.05,0.52,0.71}{\textbf{#1}}}
\newcommand\PYGaW[1]{\textcolor[rgb]{0.25,0.44,0.63}{#1}}
\newcommand\PYGaT[1]{\textcolor[rgb]{0.50,0.00,0.50}{\textbf{#1}}}
\newcommand\PYGaU[1]{\textcolor[rgb]{0.00,0.44,0.13}{#1}}
\newcommand\PYGaJ[1]{\textcolor[rgb]{0.25,0.44,0.63}{#1}}
\newcommand\PYGaK[1]{\textcolor[rgb]{0.02,0.16,0.49}{#1}}
\newcommand\PYGaH[1]{\fcolorbox[rgb]{1.00,0.00,0.00}{1,1,1}{#1}}
\newcommand\PYGaI[1]{\textcolor[rgb]{0.56,0.13,0.00}{#1}}
\newcommand\PYGaN[1]{\textcolor[rgb]{0.05,0.52,0.71}{\textbf{#1}}}
\newcommand\PYGaO[1]{\textcolor[rgb]{0.78,0.36,0.04}{\textbf{#1}}}
\newcommand\PYGaL[1]{\textcolor[rgb]{0.73,0.73,0.73}{#1}}
\newcommand\PYGaM[1]{\textcolor[rgb]{0.00,0.44,0.13}{#1}}
\newcommand\PYGaB[1]{\textcolor[rgb]{0.00,0.25,0.82}{#1}}
\newcommand\PYGaC[1]{\textcolor[rgb]{0.33,0.33,0.33}{\textbf{#1}}}
\newcommand\PYGaA[1]{\textcolor[rgb]{0.00,0.44,0.13}{#1}}
\newcommand\PYGaF[1]{\textcolor[rgb]{1.00,0.00,0.00}{#1}}
\newcommand\PYGaG[1]{\textcolor[rgb]{0.73,0.38,0.84}{#1}}
\newcommand\PYGaD[1]{\textcolor[rgb]{0.25,0.50,0.56}{\textit{#1}}}
\newcommand\PYGaE[1]{\textcolor[rgb]{0.63,0.00,0.00}{#1}}
\newcommand\PYGbg[1]{\textcolor[rgb]{0.44,0.63,0.82}{\textit{#1}}}
\newcommand\PYGbe[1]{\textcolor[rgb]{0.25,0.44,0.63}{#1}}
\newcommand\PYGbd[1]{\textcolor[rgb]{0.00,0.44,0.13}{\textbf{#1}}}
\newcommand\PYGbh[1]{\textcolor[rgb]{0.00,0.44,0.13}{\textbf{#1}}}
\begin{document}

\maketitle
\tableofcontents
\hypertarget{--doc-index}{}


Contents:

\resetcurrentobjects
\hypertarget{--doc-introduction}{}

\chapter{Introduction}

Baruwa (swahili for letter or mail) is a \href{http://mailwatch.sf.net/}{mailwatch} inspired
web 2.0 \href{http://www.mailscanner.info/}{MailScanner} front-end.

It provides an easy to use interface for users to view details of messages processed by
MailScanner as well as perform operations such as releasing quarantined messages,
spam learning,whitelisting and blacklisting addresses etc. Baruwa has implemented web 2.0
features (AJAX) where deemed fit, graphing is also implemented on the client side using SVG.

It also provides reporting functionality with an easy to use query builder, results can be
displayed as message lists or graphed as colorful and pretty graphs. The goal of the initial
version will be to address the various quarks that exist in mailwatch at the moment. Baruwa
will initially be a drop in replacement for mailwatch, future releases will however break
compatibility with mailwatch. This enables easing of end users in to the new interface.

Baruwa is written in Python using the Django Framework and MySQL for storage, it is released
under the GPLv2


\section{Features}
\begin{itemize}
\item {} 
AJAX refreshed recent messages listing

\item {} 
Detailed message view with AJAX enabled message processing (quarantine release/delete,spam learning,white/black listing)

\item {} 
AJAX powered Full/Quarantine messages listings

\item {} 
Reporting view with AJAX enabled query builder

\item {} 
Interactive SVG graphs

\item {} 
Multi user profiles

\item {} 
User profile aware white/blacklist management

\item {} 
REST API

\item {} 
Easy plug-in authentication to external authentication systems (LDAP,SQL,etc)

\item {} 
Works both with and without Javascript enabled (some features will have degraded performance)

\end{itemize}


\section{Screenshots}

A \href{http://www.flickr.com/photos/kissandrew/sets/72157623453063688/show/}{Slideshow} can be viewed
on Flickr

\resetcurrentobjects
\hypertarget{--doc-requirements}{}

\chapter{Requirements}


\section{Baruwa requirements}
\begin{itemize}
\item {} 
\href{http://www.python.org/}{Python} \textgreater{}= 2.4

\item {} 
\href{http://www.djangoproject.com/}{Django} \textgreater{}= 1.1.1

\item {} 
\href{http://mysql-python.sourceforge.net/}{MySQLdb} \textgreater{}= 1.2.1p2

\item {} 
\href{http://www.maxmind.com/app/python}{GeoIP}

\item {} 
\href{http://software.inl.fr/trac/wiki/IPy}{iPy}

\item {} 
Any Web server that can run Django (\href{http://httpd.apache.org/}{Apache}/\href{http://code.google.com/p/modwsgi/}{mod\_wsgi} recommended)

\item {} 
\href{http://www.mysql.com}{MySQL}

\end{itemize}


\section{MailScanner requirements}
\begin{itemize}
\item {} 
\href{http://dbi.perl.org/}{DBI}

\item {} 
\href{http://search.cpan.org/dist/DBD-mysql/}{DBD-MySQL}

\end{itemize}

\resetcurrentobjects
\hypertarget{--doc-install}{}

\hypertarget{install}{}\chapter{Installation}


\section{Source installation}

Download the \href{http://www.topdog.za.net/baruwa\#downloads}{Baruwa} source
and untar it:

\begin{Verbatim}[commandchars=@\[\]]
@PYGaD[@# tar xzvf baruwa-@textless[]version@textgreater[].tar.gz]
@PYGaD[@# cd baruwa-@textless[]version@textgreater[]]
\end{Verbatim}

Make sure you have the required packages installed as well as a working
MailScanner instalation then proceed. The following commands should be
run as a privileged user.

\textbf{Create the database}:

\begin{Verbatim}[commandchars=@\[\]]
@PYGaD[@# mysql -p @textless[] extras/baruwa-create.sql]
\end{Verbatim}

\textbf{Create a Mysql user for baruwa}

Run the command from the mysql prompt:

\begin{Verbatim}[commandchars=@\[\]]
mysql@textgreater[] GRANT ALL ON baruwa.* TO baruwa@PYGZat[]localhost IDENTIFIED BY '@textless[]password@textgreater[]';
mysql@textgreater[] flush privileges;
\end{Verbatim}

\textbf{Edit and install MailWatch.pm}

Edit extras/MailWatch.pm and set the variables \$db\_user to `baruwa' and
\$db\_pass to the password you set above. Then move extras/MailWatch.pm
to your MailScanner custom functions directory which is
/usr/lib/MailScanner/MailScanner/CustomFunctions or
/opt/MailScanner/lib/MailScanner/MailScanner/CustomFunctions

\textbf{Create an Admin user}

Run the commands from the mysql prompt:

\begin{Verbatim}[commandchars=@\[\]]
@# mysql baruwa -u baruwa -p
Enter password: ******
mysql@textgreater[] INSERT INTO users (username,password,type) VALUES ('@textless[]username@textgreater[]',md5('@textless[]password@textgreater[]'),'@textless[]name@textgreater[]','A');
\end{Verbatim}

\textbf{Install Baruwa}

Run:

\begin{Verbatim}[commandchars=@\[\]]
@PYGaD[@# python setup.py install]
\end{Verbatim}

\textbf{Configure the Baruwa settings}

Edit the settings.py file which will be installed in the ``baruwa''
directory inside your Python's site-packages directory, which is
located where ever your Python installation lives. Some places
to check are:
\begin{itemize}
\item {} 
/usr/lib/python2.4/site-packages (Unix, Python 2.4)

\item {} 
/usr/lib/python2.6/site-packages (Unix, Python 2.6)

\item {} 
/opt/local/lib/python2.4/site-packages (MacOSX, Ports Python 2.4)

\end{itemize}

Set the following options:

\begin{Verbatim}[commandchars=@\[\]]
DATABASE@_NAME @PYGbf[=] @PYGad[']@PYGad[baruwa]@PYGad[']
DATABASE@_USER @PYGbf[=] @PYGad[']@PYGad[baruwa]@PYGad[']
DATABASE@_PASSWORD @PYGbf[=] @PYGad[']@PYGad[@textless[]password@textgreater[]]@PYGad[']
DATABASE@_HOST @PYGbf[=] @PYGad[']@PYGad[localhost]@PYGad[']
\end{Verbatim}

If your MailScanner config file is not located in the standard
location (/etc/MailScanner/MailScanner.conf) then edit the
baruwa\_settings.py file which is in the same directory as the
settings.py file and set:

\begin{Verbatim}[commandchars=@\[\]]
MS@_CONFIG @PYGbf[=] @PYGad[']@PYGad[/etc/MailScanner/MailScanner.conf]@PYGad[']
\end{Verbatim}

To the correct file name.

\textbf{Setup Web server}
\begin{quote}

\textbf{Apache/Mod\_WSGI}

Use the sample configuration provided (extras/baruwa-mod\_wsgi.conf)
as a template. Copy to your apache configuration directory usually
/etc/httpd/conf.d

Make sure that your apache is configured for name based virtual
hosting such that you can run other sites on the same box if you
wish to.

Edit /etc/httpd/conf.d/baruwa-mod\_wsgi.conf and set ServerName to
the hostname you will use to access baruwa

Restart apache for the configuration to take effect.:

\begin{Verbatim}[commandchars=@\[\]]
@PYGaD[@# /etc/init.d/httpd reload]
\end{Verbatim}

\textbf{Apache/Mod\_python}

TODO

\textbf{Lighttpd}

TODO

\textbf{Nginx}

TODO
\end{quote}

\textbf{Setup MailScanner}

Stop MailScanner:

\begin{Verbatim}[commandchars=@\[\]]
@PYGaD[@# /etc/init.d/MailScanner stop]
\end{Verbatim}

Next edit the MailScanner config file /etc/MailScanner/MailScanner.conf,
you need to make sure that the following options are set:

\begin{Verbatim}[commandchars=@\[\]]
Quarantine User = exim (Or what ever your "Run As User" is set to)
Quarantine Group = apache (or your webserver user if not apache)
Quarantine Permissions = 0660
Quarantine Whole Message = yes
Quarantine Whole Message As Queue Files = no
Detailed Spam Report = yes
Include Scores In SpamAssassin Report = yes
Always Looked Up Last = @&MailWatchLogging
\end{Verbatim}

To actually quarantine and later process messages with in baruwa, set
`store' as one of your keywords for the ``Spam Actions'' and
``High Scoring Spam Actions'' MailScanner options

\textbf{Integrate SQL Blacklist/Whitelist}

Edit extras/SQLBlackWhiteList.pm and set the variables \$db\_user to `baruwa' and
\$db\_pass to the password you set above. The move extras/SQLBlackWhiteList.pm to
your MailScanner custom functions directory which is
/usr/lib/MailScanner/MailScanner/CustomFunctions or
/opt/MailScanner/lib/MailScanner/MailScanner/CustomFunctions

Next edit the MailScanner config file /etc/MailScanner/MailScanner.conf, and
set the following options:

\begin{Verbatim}[commandchars=@\[\]]
Is Definitely Not Spam = @&SQLWhitelist
Is Definitely Spam = @&SQLBlacklist
\end{Verbatim}

Start up MailScanner:

\begin{Verbatim}[commandchars=@\[\]]
@PYGaD[@# /etc/init.d/MailScanner start]
\end{Verbatim}

\textbf{Enjoy Baruwa}

Point your browser to \href{http://hostname\_used}{http://hostname\_used} login with admin user and password
and start working. You can now use the interface to add users and process
messages, etc etc.


\section{Distribution / OS installation}
\begin{itemize}
\item {} 
\emph{Baruwa on Centos/RHEL}.

\item {} 
Fedora

\item {} 
Debian

\end{itemize}

\resetcurrentobjects
\hypertarget{--doc-migration}{}

\chapter{Migrate from Mailwatch}


\section{Migration using source}

Download the \href{http://www.topdog.za.net/baruwa\#downloads}{Baruwa} source
and untar it:

\begin{Verbatim}[commandchars=@\[\]]
@PYGaD[@# tar xzvf baruwa-@textless[]version@textgreater[].tar.gz]
@PYGaD[@# cd baruwa-@textless[]version@textgreater[]]
\end{Verbatim}

\textbf{Update your mailwatch database}:

\begin{Verbatim}[commandchars=@\[\]]
@PYGaD[@# mysql -p @textless[]mailwatch@_database@textgreater[] @textless[] extras/baruwa-update.sql]
\end{Verbatim}

\textbf{Set the privileges for new tables}:

\begin{Verbatim}[commandchars=@\[\]]
mysql@textgreater[] GRANT ALL ON @textless[]mailwatch@_database@textgreater[].* TO @textless[]mailwatch@_user@textgreater[]@PYGZat[]localhost IDENTIFIED BY '@textless[]mailwatch@_password@textgreater[]';
mysql@textgreater[] flush privileges;
\end{Verbatim}

\textbf{Install Baruwa}

Run:

\begin{Verbatim}[commandchars=@\[\]]
@PYGaD[@# python setup.py install]
\end{Verbatim}

\textbf{Configure the Baruwa settings}

Edit the settings.py file which will be installed in the ``baruwa''
directory inside your Python's site-packages directory, which is
located where ever your Python installation lives. Some places
to check are:
\begin{itemize}
\item {} 
/usr/lib/python2.4/site-packages (Unix, Python 2.4)

\item {} 
/usr/lib/python2.6/site-packages (Unix, Python 2.6)

\item {} 
/opt/local/lib/python2.4/site-packages (MacOSX, Ports Python 2.4)

\end{itemize}

Set the following options:

\begin{Verbatim}[commandchars=@\[\]]
DATABASE@_NAME @PYGbf[=] @PYGad[']@PYGad[@textless[]mailwatch@_database@textgreater[]]@PYGad[']
DATABASE@_USER @PYGbf[=] @PYGad[']@PYGad[@textless[]mailwatch@_user@textgreater[]]@PYGad[']
DATABASE@_PASSWORD @PYGbf[=] @PYGad[']@PYGad[@textless[]mailwatch@_password@textgreater[]]@PYGad[']
DATABASE@_HOST @PYGbf[=] @PYGad[']@PYGad[localhost]@PYGad[']
\end{Verbatim}

If your MailScanner config file is not located in the standard
location (/etc/MailScanner/MailScanner.conf) then edit the
baruwa\_settings.py file which is in the same directory as the
settings.py file and set:

\begin{Verbatim}[commandchars=@\[\]]
MS@_CONFIG @PYGbf[=] @PYGad[']@PYGad[/etc/MailScanner/MailScanner.conf]@PYGad[']
\end{Verbatim}

\textbf{Setup Web server}
\begin{quote}

\textbf{Apache/Mod\_WSGI}

Use the sample configuration provided (extras/baruwa-mod\_wsgi.conf)
as a template. Copy to your apache configuration directory usually
/etc/httpd/conf.d

Make sure that your apache is configured for name based virtual
hosting such that you can run other sites on the same box if you
wish to.

Edit /etc/httpd/conf.d/baruwa-mod\_wsgi.conf and set ServerName to
the hostname you will use to access Baruwa

Restart apache for the configuration to take effect.:

\begin{Verbatim}[commandchars=@\[\]]
@PYGaD[@# /etc/init.d/httpd reload]
\end{Verbatim}

\textbf{Apache/Mod\_python}

TODO

\textbf{Lighttpd}

TODO

\textbf{Nginx}

TODO
\end{quote}

\textbf{Recreate your White/Black lists}
The Baruwa and Mailwatch lists formats are not compartible so just redo your
lists in the easy to use baruwa interface.

\textbf{Enjoy Baruwa}

Point your browser to \href{http://hostname\_used}{http://hostname\_used} login with admin user and password
and start working. You can now use the interface to add users and process
messages, etc etc.


\section{Distribution / OS migration}
\begin{itemize}
\item {} 
\emph{Migration on Centos/RHEL}.

\item {} 
Fedora

\item {} 
Debian

\end{itemize}

\resetcurrentobjects
\hypertarget{--doc-users}{}

\chapter{User Documentation}


\section{Interface primer}

TODO


\section{Whitelists/Blacklists}

TODO


\section{User accounts}

TODO


\section{Reports}

TODO


\section{Quarantine}

TODO


\renewcommand{\indexname}{Module Index}
\printmodindex
\renewcommand{\indexname}{Index}
\printindex
\end{document}
